%%% 特別研究報告書サンプル
\documentclass[dvipdfmx]{ampbt_nomag}

%%% クラスオプション:
%%% chapter: \chapterコマンドを使用可能にする(jsbook (report) を使う).
%%% その他 jsclasses に指定可能なオプションが指定できます(そのまま渡される).

%%% 題目 %%%%%%%%%%%%%%%%%%%%%%%%%%%%%%%%%%%%%%%%%%%%%%%%%%%%%%%%%%%%%%%%%%%%%%%%
\title{模倣学習による
}     % 題目1行目
      {修士論文生成手法の検討}                             % 題目2行目
      {}                                             % 題目3行目
%%% 指導教員 %%%%%%%%%%%%%%%%%%%%%%%%%%%%%%%%%%%%%%%%%%%%%%%%%%%%%%%%%%%%%%%%%%%%
\supervisors{森本淳}{教授}    % 指導教員1人目 {氏名}{職名}
            {}{}    % 指導教員2人目 {氏名}{職名}
            {}{}                % 指導教員3人目 {氏名}{職名}
%%% 入学年月 %%%%%%%%%%%%%%%%%%%%%%%%%%%%%%%%%%%%%%%%%%%%%%%%%%%%%%%%%%%%%%%%%%%%
\entrancedate{2018}{4}          % {年}{月}
%%% 著者氏名 %%%%%%%%%%%%%%%%%%%%%%%%%%%%%%%%%%%%%%%%%%%%%%%%%%%%%%%%%%%%%%%%%%%%
\author{修論}{一郎}             % {姓}{名}
%%% 提出日 %%%%%%%%%%%%%%%%%%%%%%%%%%%%%%%%%%%%%%%%%%%%%%%%%%%%%%%%%%%%%%%%%%%%%%
\submissiondate{2022}{1}{28}    % {年}{月}{日}
%%% 背表紙の幅 %%%%%%%%%%%%%%%%%%%%%%%%%%%%%%%%%%%%%%%%%%%%%%%%%%%%%%%%%%%%%%%%%%
\setlength{\wdspine}{15mm}
%%% 背表紙の出力枚数 %%%%%%%%%%%%%%%%%%%%%%%%%%%%%%%%%%%%%%%%%%%%%%%%%%%%%%%%%%%%
\def\numberofspines{1}
%%% 摘要 %%%%%%%%%%%%%%%%%%%%%%%%%%%%%%%%%%%%%%%%%%%%%%%%%%%%%%%%%%%%%%%%%%%%%%%%
\abstract{%
}
%%% パッケージの読み込みや自分用のマクロの定義 %%%%%%%%%%%%%%%%%%%%%%%%%%%%%%%%%%
\usepackage{amsmath,amssymb}
\usepackage{algorithmic}
\usepackage{algorithm}
\usepackage{graphicx}
\usepackage[subrefformat=parens]{subcaption}


%%%\usepackage{bxpapersize} %%%消すべき?
\newcommand{\rme}{\mathrm{e}}
\renewcommand{\bfdefault}{bx}


%%% 出力の制御 %%%%%%%%%%%%%%%%%%%%%%%%%%%%%%%%%%%%%%%%%%%%%%%%%%%%%%%%%%%%%%%%%%

%%% 本文を出力しない場合,次の行のコメントを外して下さい.
%%\outputbodyfalse

%%% 末尾に表紙,背表紙を出力しない場合,次の行のコメントを外して下さい.
%% \outputcoverfalse

%%% 末尾に提出用摘要を出力しない場合,次の行のコメントを外して下さい.
%% \outputabstractforsubmissionfalse

%%% ampbt.cls では表紙等の作成のために geometry パッケージを使用しているため,本文
%%% のレイアウトを変えるために \usepackage[...]{geometry} とすると Option clash が
%%% 発生します.何らかの理由で本文のレイアウトを変更したい場合は \geometry{...} を
%%% 使用して下さい.
%%% また,jsclasses を使用しているため,例えば 3cm を指定したい場合は 3truecm と書
%%% く必要があります.
%% \geometry{hmargin=3truecm,vmargin=2truecm}

\begin{document}
\ifoutputbody
%%% 中表紙,摘要,目次 %%%%%%%%%%%%%%%%%%%%%%%%%%%%%%%%%%%%%%%%%%%%%%%%%%%%%%%%%%
\makeinsidecover                % 中表紙
\makeabstract                   % 摘要
\maketoc                        % 目次
\setcounter{page}{1}            % 本文のページ番号を1から始める
%%% 本文 %%%%%%%%%%%%%%%%%%%%%%%%%%%%%%%%%%%%%%%%%%%%%%%%%%%%%%%%%%%%%%%%%%%%%%%%
\section{序論}\label{sec-intro}		% 本文の開始
人間の手を模倣した多指ハンドロボットの実用化は社会的に大きな意義がある.人間の日常生活において手で物体を操るという動作は必要不可欠であるが,多指ハンドロボットが実現することでこれらの動作を代替することができる.このような多指ハンドロボットには,多種多様な形状を持つ物体を操る能力が要求される.これまで様々な多指ハンドロボットの開発が進められてきたものの,多指ハンドロボットを人間のように器用に動かす制御手法については依然として開発が進んでおらず実用化とは程遠い状況にある.\\
多指ハンドロボットの制御手法の開発においては,多指ハンドロボットを用いた物体操作タスクは接触が多くモデル化が困難てある点が特に問題となる.このような課題を解決するため,多指ハンドロボット強化学習力学モデルの存在を陽に仮定しないモデルフリー強化学習が存在する.モデルフリー強化学習では環境とのインタラクションに



	-多指ハンドロボットの物体操作タスクは接触が多くモデル化が困難.\\
	-強化学習が一般的に用いられる.\\
	-強化学習ではマルチタスクが困難で,ハンドロボットの特徴が生かせていない.\\
従来の研究との差分\\
・従来はハイレベルのマルチタスクが多かったが,ローレベルの制御でマルチタスクを目指す.\\
・Spirlとの差分\\
本研究の新規性\\
	まだ\\

\section{関連研究}\label{sec-related_papers}
マルチタスク強化学習\\
Contact-Richなタスクの制御\\

\section{手法}\label{sec-method}
・強化学習について\\
・SACについて
・スキルを用いた強化学習\\

\section{実験}\label{sec-experiment}
・実験のセットアップ\\
	・ロボットの概要\\
	・シミュレーション環境\\
	・実機の設定\\
・実験設定0(放物線追従タスク)\\
・実験設定1(同じタスクで)\\
・実験設定2(転移)\\
・実験設定3(Sim2Real)\\

\section{議論}\label{sec-discussion}

\section{結論}\label{sec-conclusion}



%%% 謝辞 %%%%%%%%%%%%%%%%%%%%%%%%%%%%%%%%%%%%%%%%%%%%%%%%%%%%%%%%%%%%%%%%%%%%%%%%
\acknowledgment
に深く感謝する.

%%% 参考文献 %%%%%%%%%%%%%%%%%%%%%%%%%%%%%%%%%%%%%%%%%%%%%%%%%%%%%%%%%%%%%%%%%%%%
\addcontentsline{toc}{section}{\refname} % 目次に参考文献を追加する.
                                         % chapter使用時は削除すること.
\bibliographystyle{junsrt} % jplain.bstの読み込み.
\bibliography{graduation_thesis}

%%% BibTeX 等を用いる場合は,上の thebibliography 環境を消してここに該当コードを
%%% 挿入すること.
%% \bibliographystyle{...}
%% \bibliography{...}

%%% 付録 %%%%%%%%%%%%%%%%%%%%%%%%%%%%%%%%%%%%%%%%%%%%%%%%%%%%%%%%%%%%%%%%%%%%%%%%
%%% 付録は不要ならば削除してよい.
\appendix


%%% 本文ここまで %%%%%%%%%%%%%%%%%%%%%%%%%%%%%%%%%%%%%%%%%%%%%%%%%%%%%%%%%%%%%%%%
\fi
\ifoutputcover
\cleardoublepage
%%% 表紙,背表紙,提出用摘要 %%%%%%%%%%%%%%%%%%%%%%%%%%%%%%%%%%%%%%%%%%%%%%%%%%%%
\makecover                      % 表紙
\makespine[\numberofspines]     % 背表紙
\fi
\ifoutputabstractforsubmission
\makeabstractforsubmission      % 提出用摘要
\fi
\end{document}
